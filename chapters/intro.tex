The Linux operating system (often referred to as the GNU/Linux operating system
as it uses a great many components sourced from the GNU project) consists of a
multitude of ordinary `userland' programs which provide basic functionality, but
ultimately what makes Linux, Linux is the \emph{kernel}.\\

The kernel is the part of the operating system which is `in charge' and in a
sense is really the only thing your computer is running. It schedules programs,
manages shared resources, handles errors, abstracts hardware and sits in the
background as essentially the aether in which everything else resides.\\

One of the shared resources an operating system must manage is its
memory. Modern systems have gigabytes of Random Access Memory (RAM) which must
be shared between processes, drivers and internal kernel data structures. The
memory management `subsystem' (simply a subdivision of the kernel) is what
does this, and what this book describes.\\

The book is based on the most recent of the kernel at the start of writing --
\textbf{Linux 5.17} -- and all code snippets and references are relative to this
version. Additionally, while I try to keep things as architecture-independent as
possible, in order to be able to be as specific as I can I will in some cases
need to focus on one architecture in particular. For reasons of both ubiquity
(on the desktop and server) and convenience, my focus will be the
\textbf{x86-64} architecture.\\
