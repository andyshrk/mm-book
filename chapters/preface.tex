Linux is the most successful operating system of all time -- running seamlessly
on billions of devices worldwide including almost \emph{all} of the
internet's infrastructure, nearly all supercomputers and a dominant market share
of mobile devices. Its success has been nothing short of astonishing.\\

Underlying all of this are core subsystems which provide the foundation on top
of which the rest of the kernel is built such as the scheduler, the virtual file
system, networking and arguably the most fundamental of all - the memory
manager.\\

This book dives into the Linux kernel's memory management subsystem in
substantial detail, describing its core algorithms, how memory is allocated and
managed, how `memory pressure' is handled (when the system runs low on memory),
how it interacts with I/O via the page cache and much else.\\

This is not intended to be a practical how-to guide on diagnosing issues with
memory, nor is it meant to be in any way exhaustive. No author, no matter their
longevity, could dream of covering a subsystem in its entirety. It does however
try to describe in as much detail as practically possible the \emph{core}
fundamentals of how things work, the concepts, data structures and algorithms
used to manage memory.\\

Having said that, I have endeavoured to provide a considerable amount of
practical information, with examples, references to procfs and sysfs data
sources for diagnosis, a break down of dmesg output when the out of memory
killer kicks in, knobs to twiddle to adjust memory management behaviour and much
else which should hopefully render it a useful reference for practical use.\\

While the book is not exhaustive, I have tried hard to describe as much as I
practically can with references to the ultimate authority on how things work --
no, not Linus, the kernel source code. Where insufficient detail is provided you
can always refer back to this, the ultimate reference.\\

The aim of this book is to help the reader understand how memory is managed in
Linux in as much detail as possible and to provide a stepping-off point for
further investigation. The aim of writing it was to do the same for me.\\

This book is aimed at both those simply curious about how this stuff works (in
the finest tradition of hacker culture) and Linux professionals wishing to gain
a deeper understanding of how their operating system handles memory.\\


The fascination that drives my interest in this area is how simultaneously
simple and incredibly complicated this part of the Linux kernel is. Something so
seemingly straightforward hides beneath it a great deal of engineering effort
replete with trade-offs and many, many moving parts. Much like mechanical
watches I get a special thrill from knowing there is so much going on to provide
something quite so fundamental.\\
